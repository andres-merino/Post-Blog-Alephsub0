% configuración.tex

% -- Paquetes adicionales
\usepackage[utf8]{inputenc}
\usepackage[T1]{fontenc}
\usepackage[spanish]{babel}
\usepackage{aleph-comandos}
\usepackage{graphicx}
\usepackage{mathpazo}
\usepackage{background}

% Geometría para Cuadrado
\usepackage[papersize={80mm,80mm},hmargin=0.5cm,vmargin=0.5cm]{geometry}
\backgroundsetup{
    scale=1.2,
    color=black,
    opacity=0.4,
    angle=0,
    hshift=-1mm,
    vshift=0mm,
    contents={\includegraphics[width=80mm]{FondoEjercicio.jpg}}
}

% Definición de color personalizado
\definecolor{colordef}{cmyk}{0.81,0.62,0.00,0.22}

% --- Comandos personalizados para simplificar el contenido ---
\newcommand{\titulo}[2][0mm]{
    \thispagestyle{empty}
    
    \vspace*{-10mm}
    \vspace*{#1}
    \color{colordef}
    \begin{center}
        {\large
        \textbf{\textsc{Ejercicio}}\\[2mm]
        \textbf{\textsc{#2}}
        }
    \end{center}
}

\newcommand{\problema}[3][\fill]{
    \vspace*{#1}
    \noindent
    \textbf{Problema #2:} #3
}

\newcommand{\logo}{
    \vspace{\fill}
    \noindent
    \hspace{\fill}
    \includegraphics[height=0.8cm]{Logos/LogoAlephsub0-02.png}
}
