\documentclass[a4,10pt]{aleph-notas}

% -- Paquetes adicionales
\usepackage{enumitem}
\usepackage{aleph-comandos}

% -- Datos
\institucion{Proyecto Alephsub0}
\asignatura{Complejos simpliciales}
\tema{Distancia de intercalado}
\autor{Andrés Merino}
\fecha{Noviembre 2025}

\fuente{montserrat}
\logouno[4.5cm]{Logos/LogoAlephsub0-02}
\definecolor{colordef}{cmyk}{0.81,0.62,0.00,0.22}

% -- Comandos adicionales
\newtheorem*{prob}{Problema}
    \tcolorboxenvironment{prob}{%
        color=colordef,recuadrost,colback=colordef!7,drop fuzzy shadow
    }
\DeclareMathOperator{\rank}{rank}
\usepackage{tikz}
\usetikzlibrary{arrows.meta}
\usetikzlibrary{calc}
\usetikzlibrary{decorations.markings}

% Paleta de colores más elegante
\definecolor{petrol}{RGB}{30,90,110}
\definecolor{granate}{RGB}{150,30,50}
\definecolor{forest}{RGB}{20,90,40}
\definecolor{azulon}{RGB}{20,60,120}

\tikzset{
  midarrow/.style={
    postaction={decorate},
    decoration={markings,mark=at position 0.55 with {\arrow{Latex}}}
  },
  Vline/.style={petrol},
  Wline/.style={granate},
  Uline/.style={azulon},
  mapline/.style={midarrow}
}

\begin{document}

\encabezado

Consideremos las siguientes definiciones:

\begin{defi}[Módulo de persistencia]
Un \emph{módulo de persistencia continuo} $(V,a)$ está conformado por una familia $V = \{V_t\}_{t\in\mathbb{R}}$ de espacios vectoriales 
junto con una familia de aplicaciones lineales $a = \{a_{s,t}: s,t\in\mathbb{R}, s\leq t \}$ tales que
\[
    \func{a_{s,t}}{V_s}{V_t}
\]
para todo $s\leq t$, que satisfacen:
\begin{enumerate}
    \item $a_{t,t} = \mathrm{id}_{V_t}$ para todo $t\in\mathbb{R}$;
    \item $a_{s,t}=a_{u,t}\circ a_{s,u}$ siempre que $s \le u \le t$.
\end{enumerate}
\end{defi}

\begin{defi}[$\varepsilon$-intercalados]
Sean $(V,a)$ y $(W,b)$ dos módulos de persistencia continuos. Decimos que $(V,a)$ y $(W,b)$ son
\emph{$\varepsilon$-intercalados} (\textit{interleaved}) si existen dos familias de aplicaciones lineales $\Phi = \{\Phi_t\}_{t\in\mathbb{R}}$ y $\Psi = \{\Psi_t\}_{t\in\mathbb{R}}$, con
\[
\Phi_t : V_t \longrightarrow W_{t+\varepsilon},
\texty
\Psi_t : W_t \longrightarrow V_{t+\varepsilon},
\]
tales que los diagramas de conmutatividad siguientes se cumplen para todo $s\le t$:
\begin{center}
\begin{tikzpicture}[thick]
    % Puntos
    \coordinate (Vs) at (0.5,1.5);
    \coordinate (Vt) at (2.5,1.5);
    \coordinate (Ws) at (1.0,0);
    \coordinate (Wt) at (3.0,0);
    % Líneas superior e inferior
    \draw[Vline] ($(Vs)+(-0.5,0)$) -- ($(Vt)+(1.0,0)$);
    \draw[Wline] ($(Ws)+(-1.0,0)$) -- ($(Wt)+(0.5,0)$);
    % Flechas
    \draw[petrol,mapline] (Vs) -- (Vt);
    \node[petrol,below] at ($(Vs)!0.5!(Vt)$) {$a_{s,t}$};
    \draw[granate,mapline] (Ws) -- (Wt);
    \node[granate,above] at ($(Ws)!0.5!(Wt)$) {$b_{s+\varepsilon,t+\varepsilon}$};
    \draw[forest,mapline] (Vs) -- (Ws);
    \node[forest,left]  at ($(Vs)!0.5!(Ws)$) {$\Phi_s$};
    \draw[forest,mapline] (Vt) -- (Wt);
    \node[forest,right] at ($(Vt)!0.5!(Wt)$) {$\Phi_t$};
    % Puntos sobre las líneas
    \fill[petrol] (Vs) circle (2pt) node[above] {$V_s$};
    \fill[petrol] (Vt) circle (2pt) node[above] {$V_t$};
    \fill[granate] (Ws) circle (2pt) node[below] {$W_{s+\varepsilon}$};
    \fill[granate] (Wt) circle (2pt) node[below] {$W_{t+\varepsilon}$};
\end{tikzpicture}
\hspace{5mm}
\begin{tikzpicture}[thick]
    % Puntos
    \coordinate (Vs) at (1.0,1.5);
    \coordinate (Vt) at (3.0,1.5);
    \coordinate (Ws) at (0.5,0);
    \coordinate (Wt) at (2.5,0);
    % Líneas superior e inferior
    \draw[Vline] ($(Vs)+(-1.0,0)$) -- ($(Vt)+(0.5,0)$);
    \draw[Wline] ($(Ws)+(-0.5,0)$) -- ($(Wt)+(1.0,0)$);
    % Flechas
    \draw[petrol,mapline] (Vs) -- (Vt);
    \node[petrol,below] at ($(Vs)!0.5!(Vt)$) {$a_{s+\varepsilon,t+\varepsilon}$};
    \draw[granate,mapline] (Ws) -- (Wt);
    \node[granate,above] at ($(Ws)!0.5!(Wt)$) {$b_{s,t}$};
    \draw[forest,mapline] (Ws) -- (Vs);
    \node[forest,left]  at ($(Ws)!0.5!(Vs)$) {$\Psi_s$};
    \draw[forest,mapline] (Wt) -- (Vt);
    \node[forest,right] at ($(Wt)!0.5!(Vt)$) {$\Psi_t$};
    % Puntos sobre las líneas
    \fill[petrol] (Vs) circle (2pt) node[above] {$V_{s+\varepsilon}$};
    \fill[petrol] (Vt) circle (2pt) node[above] {$V_{t+\varepsilon}$};
    \fill[granate] (Ws) circle (2pt) node[below] {$W_s$};
    \fill[granate] (Wt) circle (2pt) node[below] {$W_t$};
\end{tikzpicture}
\hspace{5mm}
\begin{tikzpicture}[thick]
    % Puntos
    \coordinate (Vs) at (1.0,1.5);
    \coordinate (Vss) at (3.0,1.5);
    \coordinate (Ws) at (2.0,0);
    % Líneas superior e inferior
    \draw[Vline] ($(Vs)+(-0.5,0)$) -- ($(Vss)+(0.5,0)$);
    \draw[Wline] ($(Ws)+(-1.5,0)$) -- ($(Ws)+(1.5,0)$);
    % Flechas
    \draw[petrol,mapline] (Vs) -- (Vss);
    \node[petrol,below] at ($(Vs)!0.5!(Vss)$) {$a_{s,s+2\varepsilon}$};
    \draw[forest,mapline] (Vs) -- (Ws);
    \node[forest,left]  at ($(Vs)!0.5!(Ws)$) {$\Phi_s$};
    \draw[forest,mapline] (Ws) -- (Vss);
    \node[forest,right] at ($(Ws)!0.5!(Vss)$) {$\Psi_{s+\varepsilon}$};
    % Puntos sobre las líneas
    \fill[petrol] (Vs) circle (2pt) node[above] {$V_s$};
    \fill[petrol] (Vss) circle (2pt) node[above] {$V_{s+2\varepsilon}$};
    \fill[granate] (Ws) circle (2pt) node[below] {$W_{s+\varepsilon}$};
\end{tikzpicture}
\hspace{5mm}
\begin{tikzpicture}[thick]
    % Puntos
    \coordinate (Ws) at (1.0,0);
    \coordinate (Wss) at (3.0,0);
    \coordinate (Vs) at (2.0,1.5);
    % Líneas superior e inferior
    \draw[Wline] ($(Ws)+(-0.5,0)$) -- ($(Wss)+(0.5,0)$);
    \draw[Vline] ($(Vs)+(-1.5,0)$) -- ($(Vs)+(1.5,0)$);
    % Flechas
    \draw[granate,mapline] (Ws) -- (Wss);
    \node[granate,above] at ($(Ws)!0.5!(Wss)$) {$b_{s,s+2\varepsilon}$};
    \draw[forest,mapline] (Ws) -- (Vs);
    \node[forest,left]  at ($(Ws)!0.5!(Vs)$) {$\Psi_s$};
    \draw[forest,mapline] (Vs) -- (Wss);
    \node[forest,right] at ($(Vs)!0.5!(Wss)$) {$\Phi_{s+\varepsilon}$};
    % Puntos sobre las líneas
    \fill[granate] (Ws) circle (2pt) node[below] {$W_s$};
    \fill[granate] (Wss) circle (2pt) node[below] {$W_{s+2\varepsilon}$};
    \fill[petrol] (Vs) circle (2pt) node[above] {$V_{s+\varepsilon}$};
\end{tikzpicture}
\end{center}

Es decir, para todo $s\le t$ se cumplen las siguientes igualdades:
\[
b_{s+\varepsilon,t+\varepsilon} \circ \Phi_s = \Phi_t \circ a_{s,t},
\quad
a_{s+\varepsilon,t+\varepsilon} \circ \Psi_s = \Psi_t \circ b_{s,t},
\quad
a_{s,t+2\varepsilon} = \Psi_{t+\varepsilon} \circ \Phi_s,
\quad
b_{s,t+2\varepsilon} = \Phi_{t+\varepsilon} \circ \Psi_s.
\]  
\end{defi}

\begin{defi}[Distancia de intercalado]
La \emph{distancia de intercalado} entre dos módulos de persistencia continuos $(V,a)$ y $(W,b)$ se define como
\[
d_{\mathrm{Int}}\big[(V,a),\,(W,b)\big]
=
\inf \{\varepsilon\ge 0 \,:\, (V,a) \text{ y } (W,b) \text{ son } \varepsilon\text{-intercalados}\,\}.
\]
\end{defi}

Demostremos que, en efecto, esta definición cumple las propiedades de una seudométrica.

\begin{prob}
    La distancia de intercalado $d_{\mathrm{Int}}$ es una seudométrica sobre la clase de módulos de persistencia continuos.
\end{prob}

\begin{proof}
Sean $(V,a)$, $(W,b)$ y $(U,c)$ módulos de persistencia continuos.
Demostraremos las tres propiedades: no negatividad, simetría y desigualdad triangular.
\begin{itemize}
    \item 
    \textbf{No negatividad.}
Por definición, el conjunto sobre el que se toma el ínfimo está contenido en $[0,+\infty)$, de modo que
\[
d_{\mathrm{Int}}\big[(V,a),(W,b)\big]\ge 0.
\]

\item
\textbf{Simetría.}
De la definicíon de $\varepsilon$-intercalado, se observa que las condiciones son completamente simétricas al intercambiar los papeles de $\Phi$ y $\Psi$, y al intercambiar $(V,a)$ con $(W,b)$.
Por tanto, si $(\Phi,\Psi)$ constituye un $\varepsilon$-intercalado de $(V,a)$ a $(W,b)$, entonces $(\Psi,\Phi)$ constituye un $\varepsilon$-intercalado de $(W,b)$ a $(V,a)$. Así,
\[
d_{\mathrm{Int}}\big[(V,a),(W,b)\big] = d_{\mathrm{Int}}\big[(W,b),(V,a)\big].
\]

\item
\textbf{Desigualdad triangular.}
Supongamos ahora que:
\begin{itemize}
    \item $(V,a)$ y $(W,b)$ son $\varepsilon$-intercalados mediante las familias $\Phi^{VW}$ y $\Psi^{WV}$;
    \item $(W,b)$ y $(U,c)$ son $\delta$-intercalados mediante las familias aplicaciones $\Phi^{WU}_t$ y $\Psi^{UW}_t$.
\end{itemize}

Construiremos un $(\varepsilon+\delta)$-intercalado entre $(V,a)$ y $(U,c)$.
Para cada $t\in\mathbb{R}$ definimos:
\[
\Phi^{VU}_t
=
\Phi^{WU}_{t+\varepsilon}\circ \Phi^{VW}_t
\texty
\Psi^{UV}_t
=
\Psi^{WV}_{t+\delta}\circ \Psi^{UW}_t.
\]
Con esto, tenemos que los diagramas siguientes conmutan para todo $s\le t$ (se van omitiendo progresivamente los subíndices de las aplicaciones lineales de cada módulo de presistencia):
\begin{center}
\def\ep{0.5}   
\def\del{0.8}    
\def\hV{3.0}
\def\hW{1.5}
\def\hU{0.0}
\def\Vleft{0.5} 
\def\Vright{2.5} 
\def\Uleft{0.2} 
\def\Uright{2.2}
\begin{tikzpicture}[thick]
    % Puntos
    \coordinate (Vs) at (\Vleft,\hV);
    \coordinate (Vt) at (\Vright,\hV);
    \coordinate (Ws) at ({\Vleft+\ep},\hW);
    \coordinate (Wt) at ({\Vright+\ep},\hW);
    \coordinate (Us) at ({\Vleft+\ep+\del},\hU);
    \coordinate (Ut) at ({\Vright+\ep+\del},\hU);
    % Líneas superior e inferior
    \draw[Vline] (0,\hV) -- (4,\hV);
    \draw[Wline] (0,\hW) -- (4,\hW);
    \draw[Uline] (0,\hU) -- (4,\hU);
    % Flechas
    \draw[petrol,mapline] (Vs) -- (Vt);
    \node[petrol,below] at ($(Vs)!0.5!(Vt)$) {\small$a_{s,t}$};
    \draw[granate,mapline] (Ws) -- (Wt);
    \node[granate,above] at ($(Ws)!0.5!(Wt)$) {\small$b_{s+\varepsilon,t+\varepsilon}$};
    \draw[azulon,mapline] (Us) -- (Ut);
    \node[azulon,above] at ($(Us)!0.5!(Ut)$) {\small$c_{s+\varepsilon+\delta,t+\varepsilon+\delta}$};
    \draw[forest,mapline] (Vs) -- (Ws);
    \node[forest,left]  at ($(Vs)!0.5!(Ws)$) {\small$\Phi^{VW}_s$};
    \draw[forest,mapline] (Vt) -- (Wt);
    \node[forest,right] at ($(Vt)!0.5!(Wt)$) {\small$\Phi^{VW}_t$};
    \draw[forest,mapline] (Ws) -- (Us);
    \node[forest,left]  at ($(Ws)!0.5!(Us)$) {\small$\Phi^{WU}_{s+\varepsilon}$};
    \draw[forest,mapline] (Wt) -- (Ut);
    \node[forest,right] at ($(Wt)!0.5!(Ut)$) {\small$\Phi^{WU}_{t+\varepsilon}$};
    % Puntos sobre las líneas
    \fill[petrol] (Vs) circle (2pt) node[above] {$V_s$};
    \fill[petrol] (Vt) circle (2pt) node[above] {$V_t$};
    \fill[granate] (Ws) circle (2pt) node[below left] {\small$W_{s+\varepsilon}$};
    \fill[granate] (Wt) circle (2pt) node[below right] {\small$W_{t+\varepsilon}$};
    \fill[azulon] (Us) circle (2pt) node[below] {$U_{s+\varepsilon+\delta}$};
    \fill[azulon] (Ut) circle (2pt) node[below] {$U_{t+\varepsilon+\delta}$};
\end{tikzpicture}
\hspace{15mm}
\begin{tikzpicture}[thick]
    % Puntos
    \coordinate (Us) at (\Uleft,\hU);
    \coordinate (Ut) at (\Uright,\hU);
    \coordinate (Ws) at ({\Uleft+\del},\hW);
    \coordinate (Wt) at ({\Uright+\del},\hW);
    \coordinate (Vs) at ({\Uleft+\del+\ep},\hV);
    \coordinate (Vt) at ({\Uright+\del+\ep},\hV);
    % Líneas superior e inferior
    \draw[Vline] (0,\hV) -- (4,\hV);
    \draw[Wline] (0,\hW) -- (4,\hW);
    \draw[Uline] (0,\hU) -- (4,\hU);
    % Flechas
    \draw[petrol,mapline] (Vs) -- (Vt);
    \node[petrol,below] at ($(Vs)!0.5!(Vt)$) {\small$a_{s+\varepsilon+\delta,t+\varepsilon+\delta}$};
    \draw[granate,mapline] (Ws) -- (Wt);
    \node[granate,above] at ($(Ws)!0.5!(Wt)$) {\small$b_{s+\delta,t+\delta}$};
    \draw[azulon,mapline] (Us) -- (Ut);
    \node[azulon,above] at ($(Us)!0.5!(Ut)$) {\small$c_{s,t}$};
    \draw[forest,mapline] (Ws) -- (Vs);
    \node[forest,left]  at ($(Ws)!0.5!(Vs)$) {\small$\Psi^{WV}_{s+\delta}$};
    \draw[forest,mapline] (Wt) -- (Vt);
    \node[forest,right] at ($(Wt)!0.5!(Vt)$) {\small$\Psi^{WV}_{t+\delta}$};
    \draw[forest,mapline] (Us) -- (Ws);
    \node[forest,left]  at ($(Us)!0.5!(Ws)$) {\small$\Psi^{UW}_s$};
    \draw[forest,mapline] (Ut) -- (Wt);
    \node[forest,right] at ($(Ut)!0.5!(Wt)$) {\small$\Psi^{UW}_t$};
    % Puntos sobre las líneas
    \fill[petrol] (Vs) circle (2pt) node[above] {$V_{s+\varepsilon+\delta}$};
    \fill[petrol] (Vt) circle (2pt) node[above] {$V_{t+\varepsilon+\delta}$};
    \fill[granate] (Ws) circle (2pt) node[below left] {\small$W_{s+\delta}$};
    \fill[granate] (Wt) circle (2pt) node[below right] {\small$W_{t+\delta}$};
    \fill[azulon] (Us) circle (2pt) node[below] {$U_s$};
    \fill[azulon] (Ut) circle (2pt) node[below] {$U_t$};
\end{tikzpicture}
\\[5mm]
\def\ep{0.8}   
\def\del{1.2}  
\begin{tikzpicture}[thick]
    % Puntos
    \coordinate (Vs) at (\Vleft,\hV);
    \coordinate (Vss) at ({\Vleft+2*\del},\hV);
    \coordinate (Vsss) at ({\Vleft+2*\ep+2*\del},\hV);
    \coordinate (Ws) at ({\Vleft+\ep},\hW);
    \coordinate (Wss) at ({\Vleft+\ep+2*\del},\hW);
    \coordinate (Us) at ({\Vleft+\ep+\del},\hU);
    % Líneas superior e inferior
    \draw[Vline] (0,\hV) -- (5,\hV);
    \draw[Wline] (0,\hW) -- (5,\hW);
    \draw[Uline] (0,\hU) -- (5,\hU);
    % Flechas
    \draw[forest,mapline] (Vs) -- (Ws);
    \node[forest,left]  at ($(Vs)!0.5!(Ws)$) {\small$\Phi^{VW}_s$};
    \draw[forest,mapline] (Ws) -- (Us);
    \node[forest,left] at ($(Ws)!0.5!(Us)$) {\small$\Phi^{WU}_{s+\varepsilon}$};
    \draw[forest,mapline] (Us) -- (Wss);
    \node[forest,right]  at ($(Us)!0.5!(Wss)$) {\small$\Psi^{UW}_{s+\varepsilon+\delta}$};
    \draw[forest,mapline] (Wss) -- (Vsss);
    \node[forest,right] at ($(Wss)!0.5!(Vsss)$) {\small$\Phi^{WV}_{s+\varepsilon+2\delta}$};
    \draw[forest,mapline] (Vss) -- (Wss);
    \node[forest,left] at ($(Vss)!0.5!(Wss)$) {\small$\Psi^{WV}_{s+\varepsilon+\delta}$};
    \draw[granate,mapline] (Ws) -- (Wss);
    \node[granate,above] at ($(Ws)!0.5!(Wss)$) {\small$b_{s+\varepsilon,s+\varepsilon+2\delta}$};
    \draw[petrol,mapline] (Vs) -- (Vss);
    \node[petrol,below] at ($(Vs)!0.5!(Vss)$) {\small$a_{s,s+2\delta}$};
    \draw[petrol,mapline] (Vss) -- (Vsss);
    \node[petrol,below] at ($(Vss)!0.5!(Vsss)$) {\small$a_{\cdot,\cdot}$};
    % Puntos sobre las líneas
    \fill[petrol] (Vs) circle (2pt) node[above] {$V_s$};
    \fill[granate] (Ws) circle (2pt) node[below left] {\small$W_{s+\varepsilon}$};
    \fill[azulon] (Us) circle (2pt) node[below] {$U_{s+\varepsilon+\delta}$};
    \fill[granate] (Wss) circle (2pt) node[below right] {\small$W_{s+\varepsilon+2\delta}$};
    \fill[petrol] (Vsss) circle (2pt) node[above] {$V_{s+2\varepsilon+2\delta}$};
    \fill[petrol] (Vss) circle (2pt) node[above] {$V_{s+2\delta}$};
\end{tikzpicture}
\hspace{10mm}
\begin{tikzpicture}[thick]
    % Puntos
    \coordinate (Us) at (\Uleft,\hU);
    \coordinate (Uss) at ({\Uleft+2*\ep},\hU);
    \coordinate (Usss) at ({\Uleft+2*\ep+2*\del},\hU);
    \coordinate (Ws) at ({\Uleft+\del},\hW);
    \coordinate (Wss) at ({\Uleft+\del+2*\ep},\hW);
    \coordinate (Vs) at ({\Uleft+\del+\ep},\hV);
    % Líneas superior e inferior
    \draw[Vline] (0,\hV) -- (5,\hV);
    \draw[Wline] (0,\hW) -- (5,\hW);
    \draw[Uline] (0,\hU) -- (5,\hU);
    % Flechas
    \draw[forest,mapline] (Us) -- (Ws);
    \node[forest,left]  at ($(Us)!0.5!(Ws)$) {\small$\Psi^{UW}_\cdot$};
    \draw[forest,mapline] (Ws) -- (Vs);
    \node[forest,left] at ($(Ws)!0.5!(Vs)$) {\small$\Psi^{WV}_{\cdot}$};
    \draw[forest,mapline] (Vs) -- (Wss);
    \node[forest,right]  at ($(Vs)!0.5!(Wss)$) {\small$\Phi^{VW}_{\cdot}$};
    \draw[forest,mapline] (Wss) -- (Usss);
    \node[forest,right] at ($(Wss)!0.5!(Usss)$) {\small$\Phi^{WU}_{\cdot}$};
    \draw[forest,mapline] (Uss) -- (Wss);
    \node[forest,left] at ($(Uss)!0.5!(Wss)$) {\small$\Psi^{WU}_{\cdot}$};
    \draw[granate,mapline] (Ws) -- (Wss);
    \node[granate,above] at ($(Ws)!0.5!(Wss)$) {\small$b_{\cdot,\cdot}$};
    \draw[azulon,mapline] (Us) -- (Uss);
    \node[azulon,below] at ($(Us)!0.5!(Uss)$) {\small$c_{\cdot,\cdot}$};
    \draw[azulon,mapline] (Uss) -- (Usss);
    \node[azulon,below] at ($(Uss)!0.5!(Usss)$) {\small$c_{\cdot,\cdot}$};
    % Puntos sobre las líneas
    \fill[azulon] (Us) circle (2pt) node[below] {$U_s$};
    \fill[granate] (Ws) circle (2pt) node[above left] {\small$W_{s+\delta}$};
    \fill[petrol] (Vs) circle (2pt) node[above] {$V_{s+\varepsilon+\delta}$};
    \fill[granate] (Wss) circle (2pt) node[above right] {\small$W_{s+2\varepsilon+\delta}$};
    \fill[azulon] (Usss) circle (2pt) node[below] {$U_{s+2\varepsilon+2\delta}$};
    \fill[azulon] (Uss) circle (2pt) node[below] {$U_{s+2\varepsilon}$};
\end{tikzpicture}
\end{center}

Concluimos que $(\Phi^{VU},\Psi^{UV})$ define un $(\varepsilon+\delta)$-intercalado entre $(V,a)$ y $(U,c)$.  
Por lo tanto:
\[
d_{\mathrm{Int}}[(V,a),(U,c)] \le \varepsilon+\delta.
\]

Tomando ínfimos sobre todos los posibles $\varepsilon$ y $\delta$ obtenemos la desigualdad triangular:
\[
d_{\mathrm{Int}}[(V,a),(U,c)]
\le
d_{\mathrm{Int}}[(V,a),(W,b)]
+
d_{\mathrm{Int}}[(W,b),(U,c)].
\]

\medskip

Puesto que se verifican no negatividad, simetría y desigualdad triangular, la función $d_{\mathrm{Int}}$ es una seudométrica.
\end{itemize}

\end{proof}

\end{document}